\documentclass{article}
\usepackage{graphicx}

\begin{document}

    \title{\textbf{Problem 5}}
    \maketitle

    \section{5.1}

    We consider a b-compliment number system with base 5 and 4 ``digits".\\
    For the representation of -1 in this number system:
    First we consider the representation of 1. Which is:\\
    $1(5 ^ 0)$:\\
    \\
    0001\\
    \\
    We choose the negative such that when the two numbers are added, the last 4 "digits" represent a 0 and we ignore the overflow.
    In this case, the number that satisfies this condition for -1 is:\\
    \\
    4444\\
    \\
    We use the same process for the representation of -8.\\
    Representation of 8:\\
    \\
    $1(5 ^ 1) + 3(5 ^ 0)$\\
    \\
    0013\\
    \\
    Number which causes an overflow equivalent to (-8):\\
    \\
    4432\\
    \\
    When adding the two numbers:\\
    \\ 
    $\ 4444$\\
    $\ 4432$\\
    \_\_\_\_\_\_\_\\
    $14431$\\
    \\
    We only take 4 of the least significant bits and ignore the overflow giving us:\\
    \\
    4431\\
    \\
    Converting the number into decimal:\\
    \\
    4431 (base 5)\\
    \\
    $4(5 ^ 3) + 4(5 ^ 2) + 3(5 ^ 1) + 1(5 ^ 0)$\\
    \\
    616 (base 10)
	\\\\
	\\\\
    \section{5.2}

    We are working with a 32 bit format of floating point numbers. This means that we have a 23 bit mantissa to store numbers.
    Thus we can store 24 bits in total after normalization.\\
    The decimal number we want to convert to a floating point binary is -273.15
    Our first step is to set the "signed bit" to 1 because our number is negative.
    Then we convert the digits before   decimal to binary:\\
    273\\
    $(2 ^ 8) + (2 ^ 4) + (2 ^ 0)$\\
    100010001\\
    \\
    Now we use an algorithm for the decimal part of our number:\\
    Since we already need 8 bits to represent the previous binary, we are left with 15 bits\\
    $0.15 -> 0.3$ (0)\\
    $0.3\  -> 0.6$ (00)\\
    $0.6\  -> 1.2$ (001)\\
    $0.2\ -> 0.4$ (0010)\\
    $0.4\  -> 0.8$ (00100)\\
    $0.8\  -> 1.6$ (001001)\\
    $0.6\  -> 1.2$ (0010011)\\
    From this point on, the pattern is recursive. Thus our binary approximation will be:\\
    001001100110011\\
    We stop at the 15th bit.\\
    The binary conversion of 273.15 is:\\
    \\
    100010001.001001100110011\\
    \\
    After normalization we get the mantissa:\\
    \\
    00010001001001100110011\\
    \\
    And the exponent: $2 ^ 8$\\
    But we add 127 to the exponent and store it as binary giving us:\\
    \\
    10000111\\
    \\
    Thus the final 32 bit floating point representation is:\\
    \\
    11000011100010001001001100110011\\
    \\
    \\
    For the approximation of the decimal fraction stored:\\
    \\
    001001100110011\\
    \\
    $(2^{-3}) + (2^{-6}) + (2^{-7}) + (2^{-10}) + (2^{-11}) + (2^{-14}) + (2^{-15})$\\
    \\
    $0.1499939$

\end{document}