\documentclass{article}
\usepackage{graphicx}

\begin{document}

    \title{\textbf{Problem 3}}
    \maketitle
    \section{\textbf{3.1}}
    Prove the following statement by induction:
    The number of elements in the power set $P(S)$ of a finite set $S$ with $n$ elements is $2^n$.
    Let us assume a set $K$ where:

    $K = \{ \forall n , n \in N, |S| = n : |P(S)| = 2^n \}$
    \\\\
    Base Case : $n = 0$
    $|P(S)| = 2^0 = 1$ because the only subset of S is an empty set.
    \\\\
    Let the new set be $Q = S \cup \{a\}$.
    Thus $|Q| = n + 1$.
    For $Q$ there are two types of subsets, $P(S)$ and $R \cup {a}, R\in P(S)$.
    Since there are $|P(S)|$ amount of $R$ we can concider,
    $|P(Q)| = |P(S)| + |R \cup {a}|$
    $|P(Q)| = 2^n + 2^n$
    $|P(Q)| = 2^n(1 + 1)$
    $|P(Q)| = 2^n . 2$
    $|P(Q)| = 2^{n+1}$

    Since the assumption holds for $n + 1$, K is inductive.
    This the assumption holds for all values of $n$.


    \section{\textbf{3.2}}
    
    a) $R = \{(a, b) | a, b \in Z \land a \neq b\}$

        For reflexivity
        $\forall a \in A, (a, a) \notin R$
        Since a and b are never equal.
        \\\\
        For symmetry:
        $\forall a, b \in A, (a, b) \notin R \Rightarrow (b, a) \notin R$
        \\\\
        Because if $(a, b) \no-tin R \Rightarrow (b, a) \notin R$ it implies $a \neq b$ and $b \neq a$ which is not true for this relation.
        \\\\
        For transitivity:
        $\forall a, b, c \in A, ((a, b) \in R \land (b, c) \in R) \Rightarrow (a, c) \notin R$
        \\\\
        Because if $a, b, c \in A, ((a, b) \in R \land (b, c) \in R) \Rightarrow (a, c) \in R$ it implies $a = b$ which is not true for this relation.
        \\\\

    b) $R = \{(a, b) | a, b \in Z \land |a - b| \leq 3 \}$
        \\\\
        For reflexivity:
        $\forall a \in A, (a, a) \in R$
        \\\\
        Because the distance between same numbers is zero, which satisfies the condition.
        \\\\
        For symmetry:
        $\forall a, b \in A, (a, b) \in R \Rightarrow (b, a) \in R$
        \\\\
        Because the operation $|a - b|$ is commutative thus the relation is symmetric as long as the distance between $a$ and $b$ is less than 3.
        \\\\
        For transitivity:
        $\forall a, b, c \in A, ((a, b) \in R \land (b, c) \in R) \Rightarrow (a, c) \notin R $

        Because even if $(a, b)$ and $(b, c)$ satisfy the relation, it does not imply that $(a, c)$ satisfies the condition.
        Example:
        $(1, 3) = 2$
        $(3, 5) = 2$
        $(1, 5) = 4$ which does not satisfy the relation.
        \\\\
    c) $R = \{(a, b) | a, b \in Z \land (a mod 10) = (b mod 10) \}$
        \\\\
        For reflexivity:
        $\forall a \in A, (a, a) \in R$
        \\\\
        Because $(a mod 10) = (a mod 10)$ is true.
        \\\\
        For symmetry:
        $\forall a, b \in A, (a, b) \in R \Rightarrow (b, a) \in R$
        \\\\
        Because the operation $(a mod 10) = (b mod 10)$ is commutative.
        \\\\
        For transitivity:
        $\forall a, b, c \in A, ((a, b) \in R \land (b, c) \in R) \Rightarrow (a, c) \in R $
        \\\\
        Because  $(a, b)$ and $(b, c)$ satisfy the relation, it implies that $(a, c)$ satisfies the condition.
        If \\\\
        $(a mod 10) = (b mod 10)$ and $(b mod 10) = (c mod 10)$
        It implies: \\\\
        $(a mod 10) = (c mod 10)$
        \\\\
        
\end{document}
