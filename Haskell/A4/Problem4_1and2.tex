\documentclass{article}
\usepackage{graphicx}

\begin{document}

    \title{\textbf{Problem 4}}
    \maketitle

    \section{4.1}

    Let $\Sigma$ be a finite set of alphabets. Then $\Sigma^*$ is the set of all words that can be created
    out of the symbols in $\Sigma$ i.e. $\Sigma^*$ is the kleene closure of $\Sigma$\\

    a):\\
    Let $\preceq \subseteq \Sigma^* \times \Sigma^* : \{p, w \in \Sigma^*, \exists q \in \Sigma^*: p \preceq w \Rightarrow  w = pq \lor w = p \}$\\
    For the relation $\preceq$ on $\Sigma^*$ to be a partial order it must be reflexive, antisymmetric and transitive on $\Sigma^*$\\

    Checking for reflexivity:\\
    for $p \preceq p$\\
    This signifies that:\\
    $p = p\epsilon$ where $\epsilon$ is an empty 	set.\\
    Which is true because we allow such cases to occour in this relation. ($w = p$)\\
    Thus the relation is reflexive.\\

    Checking for antisymmetry:\\
    if $(p, q) \in \preceq q = pa$\\
    if $(q, p) \in \preceq p = qb$\\

    The only way this can be true is if $a, b$ is $\emptyset$ and $p = q$.\\
    $p = q$ is allowed for this relation. Thus The relation is antisymmetric.\\

    Checking for transitivity:\\

    for $(p, q) \in \preceq q = pa$\\
    for $(q, r) \in \preceq r = qb$\\
    if $(p, q) \in \preceq \land (q, r) \in \preceq$\\
    $r = qb$\\
    $r = (pa)b$\\
    $r = pab$\\
    Thus $(p, r) \in \preceq^*$

    b)\\
    Let $\prec \subset \Sigma^* \times \Sigma^* : \{p, w \in \Sigma^*, \exists q \in \Sigma^* : p \prec w \Rightarrow  w = pq \lor w \neq p \}$\\
    For the relation $\prec$ on $\Sigma^*$ to be a partial order it must be reflexive, antisymmetric and transitive on $\Sigma^*$\\
    
    Checking for reflexivity:\\
    for $p \preceq p$\\
    This signifies that:\\
    $p = pa$\\
    where $a$ is $\emptyset$
    Which is false because we do not allow such cases to occour in this relation. ($w \neq p$)\\
    Thus the relation is ireflexive.\\

    Checking for asymmetry:\\

    if $(p, q) \in \preceq q = pa$\\
    if $(q, p) \in \preceq p = qb$\\

    The only way this can be true is if $a, b$ is $\emptyset$ and $p = q$.
    but $p = q$ is explicitely not allowed for this relation. Thus The relation is asymmetric.

    Checking for transitivity:\\

    for $(p, q) \in \prec q = pa$\\
    for $(q, r) \in \prec r = qb$\\
    if $(p, q) \in \prec \land (q, r) \in \prec$\\
    $r = qb$\\
    $r = (pa)b$\\
    $r = pab$\\
    Thus $(p, r) \in \prec^*$ 
	\\\\
    c)\\
    The relation $\preceq$ is not total because for $(a, b) \in \preceq$ and $(b, a) \in \preceq$ if and only if $a = b$
    The relation does not allow for all other cases of $a$ and $b$ thus it is not total.

    The relation $\prec$ is not total because it does not allow for the case where $a = b\epsilon$ where $\epsilon$ is an empty set.    



    \section{4.2}

    Let $A$, $B$ and $C$ be sets and let $f : A \rightarrow B$ and $g : B \rightarrow C$ be two functions

    a)

    For $g\circ f$ if $f$ is not injective, it means that:
    $f(a) = f(b)$\\
    which implies that $g\circ f(a) = g\circ f(b)$\\
    Thus $g\circ f$ is not injective thus it is also not bijective.\\


    For $g\circ f$ if $g$ is not surjective, it means that:\\
    $\forall a \not \exists g\circ f(a)$\\
    Because if $f(a)$ exists for $a$, if $g$ is not\\ surjective then $\exists y$ such that $y = f(a) \not \exists g(y)$\\

    b)\\
    Let $f : A \rightarrow B \{\forall a \in A, a = f(a)\}$\\
    Let $g : B \rightarrow C \{\forall b \in B, b^2 = g(b)\}$\\
    $g\circ f(a) = g(f(a))$\\
    $g\circ f(a) = g(b)$\\
    $g\circ f(a) = b^2$\\

    Here, $f$ is injective and $g$ is surjective  but $g\circ f$ is not a bijective.

    c)\\
    Let $f : A \rightarrow B \{\forall a \in A, \sqrt{a} = f(a)\}$\\
    Let $g : B \rightarrow C \{\forall b \in B, b^2 = g(b)\}$\\
    $g\circ f(a) = g(f(a))$\\
    $g\circ f(a) = g(\sqrt{a})$\\
    $g\circ f(a) = (\sqrt{a})^2$\\
    $g\circ f(a) = a$\\

    Here $f$ is not surjective and $g$ is not injective but $g\circ f$ is bijective. 

\end{document}