\documentclass{article}
\usepackage{graphicx}

\begin{document}
	
    \title{\textbf{Problem 2}}
    \maketitle
    \section{\textbf{Problem 2.1}}
        If a natural number $n$ is not divisible by $3$, then the same is true for $15$.
        
        Predicate A: $\frac{n}{3}$ is a natural number
        
        Predicate B: $\frac{n}{15}$ is a natural number
        
        Theorem: If A is not true, it implies B is also not true.
        
        Proof:
            We prove the theorem by contra positive i.e.
        
            If $\neg B \longrightarrow \neg A$

            $y$ = $\frac{n}{15}$ $y \in N$

            $y$ = $\frac{n}{3 * 5}$

            We can see that if $\neg B$ is true, $\neg A$ is also true.

            Thus $A \longrightarrow B$

    \section{\textbf{Problem 2.2}}
        If $n \in N , n \leq 1$ prove that:
        
        $ 1^2 + 3^2 + 5^2 + ...(2n -1)^2 = \frac{2n(2n + 1)(2n -1)}{6} $

        Theorem: 

        $ 1^2 + 3^2 + 5^2 + ...(2n -1)^2 = \frac{2n(2n + 1)(2n -1)}{6} \forall n, n \leq 1, n \in N$

        Proof:

            Let us take a set of values $K$ such that:

            $K : \{ \forall n, n \leq 1, n \in N \mid 1^2 + 3^2 + 5^2 + ...(2n -1)^2 = \frac{2n(2n + 1)(2n -1)}{6}\}$

            For $K$ to represent all $N$ , it must be inductive. i.e.

            $n \in K \land s(n) \in K $

            $s(n) = n + 1$

            for $s(n)$,
            
                $ 1^2 + 3^2 + 5^2 + ...(2n -1)^2 + (2(n + 1)-1)^2 = \frac{2(n + 1)(2(n + 1) + 1)(2(n + 1) -1)}{6} $

                From our initial assumption of $n$,
                
                $\frac{2n(2n + 1)(2n -1)}{6} + (2(n + 1)-1)^2 = \frac{2(n + 1)(2(n + 1) + 1)(2(n + 1) -1)}{6} $

                $\frac{2n(2n + 1)(2n - 1) + 6(2n + 1)(2n + 1)}{6}  =  \frac{2n + 2)(2n + 3)(2n + 1)}{6} $
                
                $\frac{2(2n + 1)\{n(2n -1) + 3(2n + 1)\}}{6} =  \frac{2n + 2)(2n + 3)(2n + 1)}{6} $
                
                $\frac{2(2n + 1)(2n^2 + 5n + 3)}{6} =  \frac{2n + 2)(2n + 3)(2n + 1)}{6} $
                
                $\frac{2(2n + 1)\{ 2n(n + 1) +3 (n + 1)\}}{6} =  \frac{2n + 2)(2n + 3)(2n + 1)}{6} $
                
                $\frac{2(2n + 1)(n + 1)(2n + 3)}{6} =  \frac{2n + 2)(2n + 3)(2n + 1)}{6} $
                
                $\frac{(2n + 2)(2n + 3)(2n + 1)}{6} =  \frac{2n + 2)(2n + 3)(2n + 1)}{6} $

                Thus $ n \in K \land s(n) \in K$, $K \subseteq N$

                Since $K$ is inductive, the theorem holds for all $N$ 
			
\end{document}